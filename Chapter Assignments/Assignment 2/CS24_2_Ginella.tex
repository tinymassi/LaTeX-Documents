\documentclass[12pt]{article}
\usepackage{calc}
\usepackage{fancyhdr}

\pagestyle{fancy}


\rhead{Massimo Ginella}

\begin{document}
	\begin{center}
		\textbf{CS24 Elementary Computer Organization} \\
		\textbf{Chapter 2 Exercises: 2.3, 2.7, 2.13, 2.18} \vspace{0.5cm}
	\end{center}
	
	
	
	
	\noindent \fbox{\textbf{2.3}} 
	[5] \textless2.2, 2.3\textgreater \ For the following C statement, write the corresponding RISC-V assembly code. Assume that the variables $f$, $g$, $h$, $i$, and $j$ are assigned to registers x$5$, x$6$, x$7$, x$28$, and x$29$ respectively. Assume that the base address of the arrays $A$ and $B$ are in registers x$10$ and x$11$, respectively. \vspace{0.15cm} \\
	\begin{center}
		
		\begin{tabular}{ |c|c|c| } 
			\hline
			Variable & Numeric Register & Human Equivalent \\
			\hline
			$f$ & x$5$ & t0 \\ 
			\hline
			$g$ & x$6$  &  t1 \\ 
			\hline
			$h$ & x$7$  &  t2 \\ 
			\hline
			$i$ & x$28$  & t3 \\
			\hline
			$j$ & x$29$  &  t4 \\ 
			\hline
			$A$ & x$10$  &  a0 \\ 
			\hline
			$B$ & x$11$  &  a1 \\
			\hline
			
		\end{tabular}
		\vspace{0.8cm} \\
		\fbox{\textbf{C Code:}} \vspace{0.3cm} \\
		
		B[8] = A[i - j]; \vspace{0.5cm} \\
		
		\fbox{\textbf{Equivalent in RISC-V:}} \vspace{0.08cm} \\
		
		\begin{flushleft}
			\hspace{4cm} sub t0, t3, t4 \hspace{.7cm}  \# f = i - j \\
			\hspace{4cm} slli t0, t0, 2 \hspace{1cm}  \# f = (i - j) * 4 \\
			\hspace{4cm} add t1, a0, t0 \hspace{.63cm}  \# g = \& A[f] \\ 
			\hspace{4cm} lw t2, 0(t1)  \hspace{1.04cm}   \# h = A[f]\\
			\hspace{4cm} sw t2, 32(a1)  \hspace{.75cm}  \# B[8] = h\\
		\end{flushleft}
		
		\end{center}
		
		
		
		
		
		\newpage
		
		
		
		
		
		\noindent \fbox{\textbf{2.7}} 
		[5] \textless2.2, 2.3\textgreater \ Translate the following C code to RISC-V. Assume that the variables $f$, $g$, $h$, $i$, and $j$ are assigned to registers x$5$, x$6$, x$7$, x$28$, and x$29$ respectively. Assume that the base address of the arrays $A$ and $B$ are in registers x$10$ and x$11$, respectively. Assume that the arrays of $A$ and $B$ are 8-byte words. \vspace{0.15cm} \\
		\begin{center}
			
			\begin{tabular}{ |c|c|c| } 
				\hline
				Variable & Numeric Register & Human Equivalent \\
				\hline
				$f$ & x$5$ & t0 \\ 
				\hline
				$g$ & x$6$  &  t1 \\ 
				\hline
				$h$ & x$7$  &  t2 \\ 
				\hline
				$i$ & x$28$  & t3 \\
				\hline
				$j$ & x$29$  &  t4 \\ 
				\hline
				$A$ & x$10$  &  a0 \\ 
				\hline
				$B$ & x$11$  &  a1 \\
				\hline
				
			\end{tabular}
			\vspace{0.8cm} \\
			\fbox{\textbf{C Code:}} \vspace{0.3cm} \\
			
			B[8] = A[i] + A[j]; \vspace{0.5cm} \\
			
			\fbox{\textbf{Equivalent in RISC-V:}} \vspace{0.08cm} \\
			
			\begin{flushleft}
				\hspace{4cm} slli t3, t3, 2 \hspace{.7cm}  \# i = i * 4 \\
				\hspace{4cm} slli t4, t4, 2 \hspace{.7cm}  \# j = j * 4 \\
				\hspace{4cm} add t0, a0, t3 \hspace{.33cm}  \# f = \& A[i]] \\ 
				\hspace{4cm} add t1, a0, t4  \hspace{.34cm}  \# g = \& A[j]] \\ 
				\hspace{4cm} lw t0, 0(t0)  \hspace{.75cm}  \# f = A[i]\\
				\hspace{4cm} lw t1, 0(t1)  \hspace{.75cm}  \# g = A[j]\\
				\hspace{4cm} add t0, t0, t1  \hspace{.43cm}   \# f = A[i] + A[j]\\
				\hspace{4cm} sw t0, 32(a1)  \hspace{.45cm}   \# B[8] = f\\
			\end{flushleft}
			
		\end{center}
		
		
		
		
		
		
		\newpage
		
		
		
		
		
		
		\noindent \fbox{\textbf{2.13}} 
		[5] \textless2.2, 2.5\textgreater Provide the instruction type and hexadecimal representation of the following instruction: \vspace{0.15cm} \\
		
		\begin{center}
			\fbox{Instruction:} \vspace{0.4cm}\\
			sw x$5$, 32(x$30$)  \vspace{0.4cm}\\
			\textbf{This is a Type-S Instruction:} \vspace{0.15cm}\\
			\begin{tabular}{ |c|c|c|c|c|c| } 
				\hline
				6 bits & 5 bits & 5 bits & 3 bits & 5 bits & 7 bits \\
				\hline
				immediate [11 : 5] & rs2 & rs1 & funct3 & immediate[4:0] & opcode \\
				\hline
			\end{tabular}
			 \vspace{0.6cm} \\
			
			\textbf{Fill in with instruction quantities:} \vspace{0.15cm}\\
			\begin{tabular}{ |c|c|c|c|c|c| } 
				\hline
				6 bits & 5 bits & 5 bits & 3 bits & 5 bits & 7 bits \\
				\hline
				32 [11 : 5] & x$5$ & x$30$ & 010 & 32 [4:0] & 0100011 \\
				\hline
			\end{tabular}
			\vspace{0.6cm} \\
			
			\textbf{Binary representation of quantities:} \vspace{0.15cm}\\
			\begin{tabular}{ |c|c|c|c|c|c| } 
				\hline
				6 bits & 5 bits & 5 bits & 3 bits & 5 bits & 7 bits \\
				\hline
				0000001 & 00101 & 11110 & 010 & 00000 & 0100011 \\
				\hline
			\end{tabular}
			\vspace{0.6cm} \\
			
			\textbf{Break binary line into sections of 4 bits for hexadecimal translation:} \vspace{0.15cm}\\
			\begin{tabular}{ |c|c|c|c|c|c|c|c| } 
				\hline
				0000 & 0010 & 0101 & 1111 & 0010 & 0000 & 0010 & 0011 \\
				\hline
				0 & 2 & 5 & F & 2 & 0 & 2 & 3 \\
				\hline
			\end{tabular}
			\vspace{0.6cm} \\
			
			\textbf{Representation of the above instruction in hexadecimal:} \vspace{0.15cm}\\
			\fbox{025F2023} \\
		\end{center}
		
		
		
		
		
		\newpage
		
		
		
		
		
		\noindent \fbox{\textbf{2.18}} 
		[10] \textless2.6\textgreater Find the shortest sequence of RISC-V instructions that extracts bits 16 down to 11 from register x$5$ and uses the value of this field to replace bits 31 down to 26 in register x$6$ without changing the other bits of registers x$5$ or x$6$. (Be sure to test your code using x$5$ = 0 and x$6$ = 0xffffffffffffffff. Doing so may reveal a common oversight.) \vspace{0.15cm} \\
		
		\begin{center}
			\begin{tabular}{ |c|c| } 
				\hline
				First Register & x5 \\
				\hline
				Second Register & x6 \\
				\hline
				Bit Mask Register & t0 \\
				\hline
			\end{tabular}
			
			\vspace{0.6cm}
			\begin{flushleft}
				\hspace{0.4cm} lui t0, 0x0000F \hspace{0.43cm} \# Load upper 20 bits of bitmask. t0 = 0x0000F000\\
				\hspace{0.4cm} addi t0, t0, 0xC00 \# Add lower 12 bits of bitmask. t0 = 0x0000FC00\\
				\hspace{0.4cm} and t0, x5, t0 \hspace{0.7cm} \# Copy x5[11 - 16] into t0 \\
				\hspace{0.4cm} slli t0, t0, 16 \hspace{0.88cm} \# Move t0[11 - 16] to t0[27 - 32] \\
				\hspace{0.4cm} slli x6, x6, 6 \hspace{0.99cm} \# Remove x6[27-32] to free up space \\
				\hspace{0.4cm} srli x6, x6, 6 \hspace{1.15cm}\# Shift bits back into original position\\
				\hspace{0.4cm} or x6, x6, t0  \hspace{0.99cm} \# Copy t0[27-32] into x6[27-32] \\
			\end{flushleft}
		\end{center}
	
\end{document}          