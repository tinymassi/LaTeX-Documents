\documentclass[12pt]{article}
\usepackage{calc}
\usepackage{ulem}
\usepackage{fancyhdr}
\usepackage{amsmath}

\pagestyle{fancy}


\rhead{Massimo Ginella}

\begin{document}
	\begin{center}
		\textbf{CS24 Elementary Computer Organization} \\
		\textbf{Chapter 5 Exercises: 5.1.1, 5.1.2, 5.1.3, 5.2.1, 5.2.2, 5.2.3, 5.7.1, 5.7.2, 5.9.1} \vspace{0.5cm}
	\end{center}
	
	\noindent \fbox{\textbf{5.1}} 
	In this exercise we look at memory locality properties of matrix computation.
	The following code is written in C, where elements within the same row are stored
	contiguously. Assume each word is a 32-bit integer. \vspace{0.3cm} \\
	
	
		
		\noindent for (I=0; I<8; I++)  \\
		\indent for (J=0; J<8000; J++)  \\
		\indent \indent A[I][J]=B[I][0]+A[J][I];  \\
		

	
	\noindent \fbox{\textbf{5.1.1}} 
	[5] \textless5.1\textgreater \ How many 32-bit integers can be stored in a 16-byte cache block? \\
	
	
	\vspace{0.2cm}
	\noindent 16bytes = 128bits  \vspace{0.1cm}\\
	$\frac{128bits}{1block} \times \frac{1int}{32bits} =$ 4 32-bit integers \vspace{0.8cm} \\
	
	
	\noindent \fbox{\textbf{5.1.2}} 
	[5] \textless5.1\textgreater \ References to which variables exhibit temporal locality? \\
	
	\vspace{0.2cm}
	\noindent The variables I and J are constantly being changed through the use of loops in this code. This means that as they are being incremented, their values are being stored in the cache and therefore they are highly likely to be accessed many more times which demonstrates the use of temporal locality. \vspace{0.8cm} \\
	
	\noindent \fbox{\textbf{5.1.3}} 
	[5] \textless5.1\textgreater \ References to which variables exhibit spatial locality? \\
	
	\vspace{0.2cm}
	\noindent A[I][J] is exhibiting spatial locality as the values stored at the arrays address in memory are close together. Because of this, as the 2D array is indexed, the cache is likely to use values that are close to the ones being accessed through the means of this array. \vspace{0.8cm} \\
	
	
	\noindent \fbox{\textbf{5.2}} 
	Caches are important to providing a high-performance memory hierarchy
	to processors. Below is a list of 32-bit memory address references, given as word
	addresses. \vspace{0.3cm} \\
	
	\noindent 3, 180, 43, 2, 191, 88, 190, 14, 181, 44, 186, 253  \vspace{0.3cm} \\
	
	\noindent \fbox{\textbf{5.2.1}} 
	[10] \textless5.3\textgreater \ For each of these references, identify the binary address, the tag,
	and the index given a direct-mapped cache with 16 one-word blocks. Also list if each
	reference is a hit or a miss, assuming the cache is initially empty. \\
	
	\begin{center}
		\begin{tabular}{ |c|c|c|l| } 
			\hline
			Index & Valid & Tag &  \\
			\hline
			0000 & 0 & - & \\
			\hline
			0001 & 0 & - & \\
			\hline
			0010 & 1 & 0000 & 2 \\
			\hline
			0011 & 1 & 0000 & 3\\
			\hline
			0100 & 1 & 1011 & 180\\
			\hline
			0101 & 1 & 1011 & 181 \\
			\hline
			0110 & 0 & - & \\
			\hline
			0111 & 0 & - & \\
			\hline
			1000 & 0101 & 88 & \\
			\hline
			1001 & 0 & - & \\
			\hline
			1010 & 1 & 1011 & 186\\
			\hline
			1011 & 1 & 0010 & 43\\
			\hline
			1100 & 1 & 0010 & 44\\
			\hline
			1101 & 1 & 1111 & 253\\
			\hline
			1110 & 1 & 0000 & \sout{180} 14 \\
			\hline
			1111 & 1 & 1011 & 191 \\
			\hline
			
		\end{tabular}
		\vspace{0.3cm} \\

		$\overset{m}{3}$ $\overset{m}{180}$ $\overset{m}{43}$ $\overset{m}{2}$ $\overset{m}{191}$ $\overset{m}{88}$ $\overset{m}{190}$ $\overset{m}{14}$ $\overset{m}{181}$ $\overset{m}{44}$ $\overset{m}{186}$ $\overset{m}{253}$  \\
		
	\end{center}
	
	\newpage
	
	\noindent \fbox{\textbf{5.2.2}} 
	[10] \textless5.3\textgreater \ For each of these references, identify the binary address, the tag,
	and the index given a direct-mapped cache with two-word blocks and a total size of 8
	blocks. Also list if each reference is a hit or a miss, assuming the cache is initially empty. \\
	
	\begin{center}
		\begin{tabular}{ |c|c|c|l| } 
			\hline
			Index & Valid & Tag &  \\
			\hline
			000 & 0 & - & \\
			\hline
			001 & 1 & 0000 & 2-3 \\
			\hline
			010 & 1 & 1011 & 180-181 \\
			\hline
			011 & 0 & - & \\
			\hline
			100 & 1 & 0101 & 88-89\\
			\hline
			101 & 1 & 1011 & \sout{42-43} 186-187 \\
			\hline
			110 & 1 & 1111 & \sout{44-45} 252-253 \\
			\hline
			111 & 1 & 0000 & \sout{190-191} 14-15\\
			\hline
			
		\end{tabular}
		\vspace{0.3cm} \\
		
		$\overset{m}{3}$ $\overset{m}{180}$ $\overset{m}{43}$ $\overset{HIT}{2}$ $\overset{m}{191}$ $\overset{m}{88}$ $\overset{HIT}{190}$ $\overset{m}{14}$ $\overset{HIT}{181}$ $\overset{m}{44}$ $\overset{m}{186}$ $\overset{m}{253}$  \vspace{0.5cm} \\
		
	\end{center}
	
		\noindent \fbox{\textbf{5.2.3}} 
	[20] \textless5.3, 5.4\textgreater \ You are asked to optimize a cache design for the given
	references. There are three direct-mapped cache designs possible, all with a total of 8
	words of data: C1 has 1-word blocks, C2 has 2-word blocks, and C3 has 4-word blocks.
	In terms of miss rate, which cache design is the best? If the miss stall time is 25 cycles,
	and C1 has an access time of 2 cycles, C2 takes 3 cycles, and C3 takes 5 cycles, which is
	the best cache design? \\
	
	\noindent \textbf{CACHE C1:}  \\
	
	\begin{center}
		\begin{tabular}{ |c|c|c|l| } 
			\hline
			Index & Valid & Tag &  \\
			\hline
			000 & 1 & 01011 & 88\\
			\hline
			001 & 0 & - & \\
			\hline
			010 & 1 & 10111 & \sout{2} 186 \\
			\hline
			011 & 1 & 00101 & \sout{3} 43 \\
			\hline
			100 & 1 & 00101 & \sout{180} 44 \\
			\hline
			101 & 1 & 11111 & \sout{181} 253 \\
			\hline
			110 & 1 & 00001 & 14 \\
			\hline
			111 & 1 & 10111 & \sout{191} 190 \\
			\hline
			
		\end{tabular}
		\\
		$\overset{m}{3}$ $\overset{m}{180}$ $\overset{m}{43}$ $\overset{m}{2}$ $\overset{m}{191}$ $\overset{m}{88}$ $\overset{m}{190}$ $\overset{m}{14}$ $\overset{m}{181}$ $\overset{m}{44}$ $\overset{m}{186}$ $\overset{m}{253}$  \vspace{0.5cm} \\
	\end{center}
	
	\noindent \textbf{CACHE C2:}  \\
	
	\begin{center}
		\begin{tabular}{ |c|c|c|l| } 
			\hline
			Index & Valid & Tag &  \\
			\hline
			00 & 1 & 01011 & 88-89\\
			\hline
			01 & 1 & 10111 & \sout{2-3} \sout{42-43} \sout{2-3} 186-187 \\
			\hline
			10 & 1 & 11111 & \sout{180-181} \sout{44-45} 252-253\\
			\hline
			11 & 1 & 00001 & \sout{190-191} 14-15\\
			\hline
			
		\end{tabular}
		\\
		$\overset{m}{3}$ $\overset{m}{180}$ $\overset{m}{43}$ $\overset{m}{2}$ $\overset{m}{191}$ $\overset{m}{88}$ $\overset{HIT}{190}$ $\overset{m}{14}$ $\overset{HIT}{181}$ $\overset{m}{44}$ $\overset{m}{186}$ $\overset{m}{253}$  \vspace{0.5cm} \\
	\end{center}
	
	\noindent \textbf{CACHE C3:}  \\
	

		\begin{tabular}{ |c|c|c|l| } 
			\hline
			Index & Valid & Tag &  \\
			\hline
			0 & 1 & 10111 & (\sout{0-1-2-3}) (\sout{40-41-42-43}) (\sout{0-1-2-3}) \\ & & & (\sout{88-89-90-91}) 184-185-186-187 \\
			\hline
			1 & 1 & 11111 & (\sout{180-181-182-183})  (\sout{188-189-190-191})\\ & & &  (\sout{12-13-14-15}) (\sout{180-181-182-183}) \\ & & & (\sout{44-45-46-47}) 252-253-254-255 \\
			\hline
			
		\end{tabular}
		
	\begin{center}
		$\overset{m}{3}$ $\overset{m}{180}$ $\overset{m}{43}$ $\overset{m}{2}$ $\overset{m}{191}$ $\overset{m}{88}$ $\overset{HIT}{190}$ $\overset{m}{14}$ $\overset{m}{181}$ $\overset{m}{44}$ $\overset{m}{186}$ $\overset{m}{253}$  \vspace{0.5cm} \\
	\end{center}
	
	\noindent Out of all the caches, cache C2 has the best performance as it has the most amount of data values that resulted in a HIT.  \vspace{0.3cm} \\
	
	
		
	\noindent The rate at which C1 misses when placing values in the cache is $\frac{12}{12} \times 100 = 100\%$. The total cycles for C1 = $total misses \times miss stall time + references \times access time = 12 \times 25 + 12 \times 2 = 324 cycles$  \\
	
	\noindent The rate at which C2 misses when placing values in the cache is $\frac{10}{12} \times 100 = 83\%$. The total cycles for C2 = $total misses \times miss stall time + references \times access time = 10 \times 25 + 12 \times 3 = 286 cycles$  \\
	
	\noindent The rate at which C3 misses when placing values in the cache is $\frac{11}{12} \times 100 = 92\%$. The total cycles for C3 = $total misses \times miss stall time + references \times access time = 11 \times 25 + 12 \times 5 = 335 cycles$  \\
	
	\noindent So, the second cache has the best performance since it requires the least amount of cycles.   \vspace{0.5cm} \\
	

	
	
	\noindent \fbox{\textbf{5.7}} 
	This exercise examines the impact of different cache designs, specifically
	comparing associative caches to the direct-mapped caches from Section 5.4. For these
	exercises, refer to the address stream shown in Exercise 5.2 \vspace{0.3cm} \\
	
	\noindent 3, 180, 43, 2, 191, 88, 190, 14, 181, 44, 186, 253  \vspace{0.3cm} \\
	
	\noindent \fbox{\textbf{5.7.1}} 
	[10] \textless5.4\textgreater \ Using the sequence of references from Exercise 5.2, show the final
	cache contents for a three-way set associative cache with two-word blocks and a total
	size of 24 words. Use LRU replacement. For each reference identify the index bits, the
	tag bits, the block off set bits, and if it is a hit or a miss. \\
	
	
	\begin{center}
		\begin{tabular}{ |c|c|c|c|c|c|c|l| } 
			\hline
			Index & Valid & Tag & Valid & Tag & Valid & Tag &  \\
			\hline
			00 & 1 & 01011 & 0 & - & 0 & - & 88-89 \\
			\hline
			01 & 1 & 00101 & 1 & 00000 & 1 & 10111 & \sout{2-3} 42-43 2-3 186-187\\
			\hline
			10 & 1 & 10110 & 1 & 00101 & 1 & 11111 & 180-181 44-45 252-253 \\
			\hline
			11 & 1 & 10111 & 1 & 00001 & 0 & - & 190-191 14-15 \\
			\hline
			
		\end{tabular}
		\vspace{0.3cm} \\
		
		$\overset{m}{3}$ $\overset{m}{180}$ $\overset{m}{43}$ $\overset{HIT}{2}$ $\overset{m}{191}$ $\overset{m}{88}$ $\overset{HIT}{190}$ $\overset{m}{14}$ $\overset{HIT}{181}$ $\overset{m}{44}$ $\overset{m}{186}$ $\overset{m}{253}$  \vspace{0.5cm} \\
	\end{center}
	
	
	
	\noindent \fbox{\textbf{5.7.2}} 
	[10] \textless5.4\textgreater \ Using the references from Exercise 5.2, show the final cache
	contents for a fully associative cache with one-word blocks and a total size of 8 words.
	Use LRU replacement. For each reference identify the index bits, the tag bits, and if it
	is a hit or a miss. \\
	
	
	\begin{center}
		\begin{tabular}{ |c|c|c|c|c|c|c|c|c| } 
			\hline
			References: & \sout{3} 181 & \sout{180} 44 & \sout{2} 186 & \sout{191} 253 & 191 & 88 & 190 & 14  \\
			\hline
			Tags: & 0xB5 & 0x2C & 0xBA & 0xFD & 0xBF & 0x58 & 0xBE & 0x0E \\
			\hline
			Valid:& 1 & 1 & 1 & 1 & 1 & 1 & 1 & 1 \\
			\hline
					
		\end{tabular}
		(I wrote the tags in hex because if they were written in binary the table wouldnt fit) \\
		\vspace{0.3cm} \\
		
		$\overset{m}{3}$ $\overset{m}{180}$ $\overset{m}{43}$ $\overset{m}{2}$ $\overset{m}{191}$ $\overset{m}{88}$ $\overset{m}{190}$ $\overset{m}{14}$ $\overset{m}{181}$ $\overset{m}{44}$ $\overset{m}{186}$ $\overset{m}{253}$  \vspace{0.5cm} \\
	\end{center}
	
	
	
	
	\noindent \fbox{\textbf{5.9}} 
	This Exercise examines the single error correcting, double error detecting (SEC/
	DED) Hamming code. \vspace{0.3cm} \\
	
	\noindent \fbox{\textbf{5.9.1}} 
	[5] \textless5.5\textgreater \ What is the minimum number of parity bits required to protect a
	128-bit word using the SEC/DED code? \\
	
	The following formula can be used to determine the minimum number of parity bits required for a 128-bit word:  \\
	
	\noindent $2^p >= p + d $ \\
	$2^p >= p + 128bits$ \\
	
	\noindent The minimum number that can be plugged into p in the expression to make it logically correct is the value 8: \\
	
	\noindent $2^8 >= 8 + 128bits$ \\
	$256 >= 136$\\
	
	\noindent So the minimum number of parity bits needed is 8. \\
	
	
\end{document}  